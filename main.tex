\documentclass{article}
\usepackage{arxiv}

\usepackage[utf8]{inputenc} % allow utf-8 input
\usepackage[T1]{fontenc} % use 8-bit T1 fonts
\usepackage{hyperref} % hyperlinks
\usepackage{url} % simple URL typesetting
\usepackage{booktabs} % professional-quality tables
\usepackage{amsfonts} % blackboard math symbols
\usepackage{nicefrac} % compact symbols for 1/2, etc.
\usepackage{microtype} % microtypography
\usepackage{lipsum} % Can be removed after putting your text content
\usepackage{multicol}

\title{A Survey of Techniques for Multitask Reinforcement Learning}

%% \date{Spring 2019} % Here you can change the date presented in the paper title
\date{} 					% Or removing it

\author{
  Laura D'Arcy\\
  School of Computer Science\\
  Cardiff University\\
  \texttt{DArcyL@cardiff.ac.uk}}

\begin{document}
\maketitle

\begin{abstract}
  In recent years there has been a growing need to have tools and techniques that are able to explain the different aspects of deep learning based models. In response there have been a wealth of different approaches and ideas about the interpretability of deep neural networks, as well as efforts to define and categorise the idea of explainability with respect to human consumers, drawing ideas from explainable AI from the 1980s, social sciences research, as well as architecturally relevant techniques from neural network models. Much of the research in interpretable deep learning has understandably been focused on explaining visual classification models, however following the steady rise in performance of deep neural networks operating on the audio modality, there is equally rising interest in building interpretable audio models.
  This survey provides an overview of the current landscape of deep neural network interpretability, applications of deep neural networks to the task of sound event detection, and finally identifies potential areas of work in order to build more interpretable deep learning systems for the task of sound event detection.
\end{abstract}

% keywords can be removed
\keywords{Deep Neural Networks \and Interpretability \and Sound Event
  Detection}
\vfill
\tableofcontents 
\newpage



\section{Introduction}

\begin{itemize}
    \item policy v off policy
    \item model based v model free
    \item discrete vs continuous
    \item other delineators?
\end{itemize}
also, open challenge: exploration
lite review section on exploration papers
\subsection{Replay Methods}\label{section:RL_Graphs}

\subsection{Multi-Task RL}\label{section:RL_Complex}

\end{document}
